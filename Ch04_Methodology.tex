\chapter{Metodología: Diseño del Sistema Híbrido CLIPS-ChatGPT/Qwen}
\section{Diseño del Sistema Basado en Reglas con CLIPS}
\subsection{Definición del Conjunto de Reglas Jerárquicas}

Especificación de las reglas organizadas por niveles de prioridad, donde las reglas de seguridad (evitación de colisiones, gestión de emergencias) tienen máxima prioridad sobre las operativas.

\subsection{Priorización de Acciones Críticas y Gestión de Emergencias}

Diseño de reglas específicas para escenarios de fallo y situaciones críticas, asegurando respuestas predecibles y seguras.

\subsection{Representación del Conocimiento y Hechos en la Base de Datos de CLIPS}

Estructuración del conocimiento del entorno doméstico en hechos CLIPS iniciales que sirven como base para el razonamiento.

\subsubsection*{Descripción de la base de conocimiento inicial incluye:}

\begin{itemize}
    \item \textit{Topología del entorno:} habitaciones, conexiones, zonas navegables.
    \item \textit{Taxonomía de objetos:} categorías, subcategorías, propiedades heredables.
    \item \textit{Relaciones espaciales:} contención, adyacencia, orientación.
    \item \textit{Capacidades del robot:} acciones posibles, restricciones físicas.
\end{itemize}

\section{Integración de ChatGPT/Qwen como Sistema Complementario}
\subsection{Funciones Asignadas al Módulo de Lenguaje Natural}
\subsection{Protocolo de Comunicación entre CLIPS y los Modelos de Lenguaje}
\subsection{Mecanismos de Seguridad y Validación para las Respuestas del LLM}

Implementación de verificaciones para asegurar que los planes generados por LLMs cumplen con constraints de seguridad y factibilidad robótica.

\section{Implementación de Planificación Dual}
\subsection{Diseño del Módulo de Planificación con ChatGPT}
\subsection{Protocolos de Prompting para Generación de Planes Robóticos}
\subsection{Mecanismo de Selección entre Planes CLIPS vs. ChatGPT}
\subsection{Validación y Simulación de Planes Antes de Ejecución}

\subsubsection*{Integración con el Robot Justina}

\begin{itemize}
    \item \textit{Procesamiento de Entrada de Voz.}
    \item \textit{Ejecución de Planes en el Entorno Físico:}
    \begin{itemize}
        \item Mecanismos para transformar los planes publicados en los tópicos de resultados en comandos ejecutables por los actuadores del robot (navegación, manipulación de objetos, etc.)
    \end{itemize}
    \item \textit{Retroalimentación y Aprendizaje Incremental.}
\end{itemize}