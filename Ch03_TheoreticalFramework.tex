\chapter{Marco Teórico y Conceptual}
\section{Arquitectura de un Sistema de Planeación de Acciones Jerárquico}
\section{Fundamentos de los Sistemas de Producción (Rule-Based Systems)}
\section{Introducción a los Modelos de Lenguaje Grande (LLMs) y ChatGPT}
\section{Integración de Traductores de Lenguaje Natural con Sistemas Basados en Reglas}
\subsection{Utilidad y Función de la Integración}
\subsection{Funcionamiento de la Integración CLIPS-Lenguaje Natural}
\subsection{Descripción de Tecnologías: CLIPS, ChatGPT y Qwen2.5-0.5B}
\section{ChatGPT como Agente de Planificación Autónomo}
\subsection{Capacidades de Planificación de LLMs vs. Sistemas Basados en Reglas}
\subsection{Arquitectura de Planificación Dual: CLIPS y ChatGPT como Motores Complementarios}
\subsection{Mecanismos de Selección y Validación de Planes}
\subsection{Ventajas y Riesgos de la Planificación con LLMs en Robótica}