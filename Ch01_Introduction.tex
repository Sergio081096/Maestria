\chapter{Introducción}

Los robots de servicio doméstico representan un campo de proyección para la inteligencia artificial y la automatización, con aplicaciones que van desde la asistencia a personas mayores hasta la gestión autónoma de entornos residenciales \cite{Murphy2019}. En este contexto, la planificación de acciones, es decir, la capacidad de un robot para descomponer objetivos de alto nivel en secuencias ejecutables, es un desafío  en entornos dinámicos y no estructurados donde interactúan humanos, objetos móviles y tareas imprevistas \cite{Russell2020}.

Los sistemas basados en reglas, como los implementados en motores de inferencia del tipo CLIPS, han demostrado ser robustos y predecibles en escenarios estructurados, como líneas de producción o laboratorios controlados \cite{Giarratano2005}. Su fortaleza radica en la transparencia del razonamiento simbólico y la capacidad de priorizar acciones críticas, como la evasión de obstáculos o la gestión de emergencias. Sin embargo, su rigidez los hace poco adaptables ante variaciones no previstas en el entorno o ante comandos expresados en lenguaje natural con alto grado de ambigüedad o complejidad \cite{Brooks1986}.

Recientemente, los modelos de lenguaje grande (LLMs), como ChatGPT, han emergido como herramientas capaces de interpretar y generar lenguaje natural, e incluso de actuar como agentes de planificación autónomos \cite{Vaswani2017}. Su flexibilidad contextual permite traducir instrucciones verbales en secuencias de acciones, complementando así la solidez de los sistemas basados en reglas. No obstante, su naturaleza probabilística y la falta de garantías formales de seguridad limitan su uso directo en aplicaciones robóticas donde la integridad física y la predictibilidad son prioritarias \cite{Shi2023}.

Esta tesis propone un \textbf{sistema híbrido de planificación de tareas} que integra un motor de reglas CLIPS con modelos de lenguaje natural del tipo ChatGPT, aprovechando las ventajas de ambos enfoques: la previsibilidad y seguridad de CLIPS, en conjunto con la adaptabilidad y capacidad de comprensión lingüística de los LLMs. El sistema se implemento y valido en el robot de servicio doméstico \textit{Justina}, desarrollado en el Laboratorio de Bio-Robótica de la UNAM, utilizando ROS 2 como marco de integración.

\section{Contexto y Motivación}

El área de la rebotica orientada al servicio doméstico ha evolucionado significativamente en la última década, impulsada por avances en percepción, planificación y interacción humano-robot \cite{Murphy2019}. Estos sistemas se han diseñado para operar en entornos no estructurados como hogares, hospitales o centros de cuidado, donde deben realizar tareas que van desde la entrega de objetos hasta la asistencia en actividades de la vida diaria. En este contexto, la capacidad de planificar acciones de manera autónoma, segura y adaptativa se convierte en un requisito fundamental.

La planificación de acciones en robótica ha sido tradicionalmente abordada mediante sistemas basados en reglas \textit{(Rule-Based Systems, RBS)}, que ofrecen un marco predecible y verificable para la toma de decisiones \cite{Russell2020}. Sin embargo, la creciente complejidad de los entornos domésticos—caracterizados por la presencia de humanos, objetos dinámicos y situaciones imprevistas—ha expuesto las limitaciones de los enfoques puramente simbólicos. Paralelamente, el reciente surgimiento de modelos de lenguaje grande \textbf{(LLMs)} ha abierto nuevas posibilidades para la interpretación de comandos en lenguaje natural y la generación de planes flexibles, aunque a menudo carentes de garantías formales de seguridad \cite{Shi2023}.

Esta tesis se desarrolla en el marco del Laboratorio de Bio-Robótica de la UNAM, utilizando como plataforma el robot de servicio Justina, un sistema modular basado en ROS 2 con capacidades avanzadas de navegación, manipulación y percepción. La motivación central del trabajo radica en la necesidad de desarrollar arquitecturas de planificación que combinen lo mejor de ambos mundos: la robustez y transparencia de los sistemas basados en reglas, y la adaptabilidad y capacidad de diálogo de los modelos de lenguaje modernos.

\section{Problemática: Sistemas Basados en Reglas en Entornos Dinámicos}

Los sistemas basados en reglas, implementados en motores de inferencia como CLIPS, se fundamentan en la lógica simbólica y el ciclo \textit{reconocer-actuar}  \cite{Giarratano2005}. Su principal ventaja es la previsibilidad: ante un conjunto de hechos y reglas bien definidos, la respuesta del sistema es determinista y explicable. Esto los hace idóneos para aplicaciones donde la seguridad y la certificación son críticas, como en entornos industriales o médicos.

No obstante, en contextos domésticos dinámicos, estos sistemas enfrentan desafíos significativos \cite{Brooks1986}:

\begin{enumerate}
	\item \textbf{Rigidez interpretativa:} No pueden procesar comandos en lenguaje natural complejo o ambiguo, como "trae el libro rojo que está cerca de la ventana".
	\item \textbf{Falta de adaptabilidad:} Las reglas deben ser definidas a priori; cualquier situación no prevista en la base de conocimiento puede llevar al fracaso o a un comportamiento no deseado.
	\item \textbf{Dificultad para gestionar la incertidumbre:} No manejan bien la información incompleta o cambiante del entorno en tiempo real.
	\item \textbf{Escalabilidad limitada:} Mantener y extender un conjunto grande de reglas para cubrir todos los escenarios posibles resulta complejo y laborioso.
\end{enumerate}

Por otro lado, los enfoques puramente basados en aprendizaje automático — especialmente los LLMs—ofrecen flexibilidad y capacidad de generalización, pero introducen riesgos como \cite{Shi2023}:
\begin{itemize}
	\item Comportamientos impredecibles o no verificables.
	\item Falta de garantías de seguridad en la ejecución física.
	\item Dependencia de grandes volúmenes de datos y recursos computacionales.
\end{itemize}

La problemática central, por tanto, es cómo diseñar un sistema de planificación que mantenga la seguridad y explicabilidad de los RBS, pero que sea lo suficientemente flexible para operar en entornos domésticos dinámicos e interactuar de manera natural con humanos.

\section{Hipótesis}

La hipótesis central de esta investigación propone la integración sinérgica de un sistema basado en reglas (CLIPS) con un modelo de lenguaje grande (ChatGPT o Qwen) puede superar las limitaciones de cada enfoque por separado, resultando en un sistema híbrido de planificación que es a la vez seguro, explicable, adaptable y capaz de entender lenguaje natural.

Esta complementariedad se articula en tres niveles:

\begin{enumerate}
	\item \textbf{Nivel de interpretación:} Los LLMs actúan como traductores de lenguaje natural a hechos estructurados, permitiendo que comandos complejos y ambiguos sean convertidos en representaciones simbólicas que CLIPS puede procesar.
	\item \textbf{Nivel de planificación:} Se implementa una arquitectura de planificación dual, donde CLIPS genera planes predecibles y seguros para tareas conocidas, mientras que los LLMs proponen soluciones flexibles para situaciones novedosas o complejas.
	\item \textbf{Nivel de validación y selección:} Un mecanismo de supervisión evalúa los planes generados por ambos motores con base en criterios de seguridad, eficiencia y contexto, seleccionando la mejor opción o combinándolas de manera segura.
\end{enumerate}

Se espera que este sistema híbrido:

\begin{itemize}
	\item Mejore la tasa de éxito en la ejecución de tareas en entornos domésticos.

	\item Reduzca el tiempo de respuesta ante comandos complejos.

	\item Mantenga un comportamiento seguro y predecible en situaciones críticas.

	\item Permita una interacción más natural e intuitiva con usuarios no expertos.
\end{itemize}

La validación de esta hipótesis se realizará mediante experimentos en simulación y con el robot físico Justina, comparando el desempeño del sistema híbrido frente a enfoques puramente basados en reglas o puramente basados en LLMs.

\section{Objetivos}

\begin{itemize}
    \item \textbf{Diseño del Sistema Híbrido:}
    \begin{itemize}
        \item Definir un conjunto jerárquico de reglas en CLIPS para priorizar acciones críticas.
        \item Especificar la arquitectura de integración entre CLIPS y modelos de lenguaje natural.
        \item Diseñar un marco de planificación dual que permita tanto a CLIPS como a ChatGPT generar y ejecutar planes.
        \item Establecer protocolos de comunicación y validación entre los módulos.
    \end{itemize}
    \item \textbf{Implementación Técnica:}
    \begin{itemize}
        \item Integrar el motor de reglas CLIPS con ROS (Robot Operating System).
        \item Desarrollar el módulo de interfaz con APIs de ChatGPT o implementación local de Qwen2.5-0.5B.
        \item Implementar capacidades de planificación autónoma en ChatGPT mediante prompting estructurado.
        \item Desarrollar un mecanismo de selección y validación entre planes generados por diferentes motores.
    \end{itemize}
    \item \textbf{Capacidades de Interacción Natural:}
    \begin{itemize}
        \item Implementar traducción de comandos en lenguaje natural a hechos estructurados de CLIPS.
        \item Desarrollar mecanismos de diálogo para resolución de ambigüedades.
        \item Crear protocolos de retroalimentación para aprendizaje incremental.
    \end{itemize}
    \item \textbf{Validación Experimental:}
    \begin{itemize}
        \item Validar el sistema en escenarios realistas con el robot físico y simulaciones.
        \item Evaluar el rendimiento mediante métricas comparativas:
        \begin{itemize}
            \item Tiempo de ejecución de tareas y tasa de éxito.
            \item Precisión en interpretación de comandos complejos.
            \item Robustez ante instrucciones ambiguas o novedosas.
        \end{itemize}
        \item Comparar el sistema híbrido vs. abordajes puramente basados en reglas.
        \item Evaluar la calidad de planes generados por CLIPS vs. ChatGPT en diferentes escenarios.
    \end{itemize}
\end{itemize}

\section{Estructura del documento}

Este documento está organizado en los siguientes capítulos que describen el sistema propuesto para el desarrollo de esta tesis, abarcando el planteamiento, desarrollo, implementación y validación del sistema híbrido de planificación. La estructura es la siguiente:

\begin{enumerate}
    \item \textbf{Capítulo 1: Introducción.} Presenta el contexto, motivación, problemática, hipótesis y objetivos de la investigación, así como la justificación del enfoque híbrido en robótica de servicio doméstico.

    \item \textbf{Capítulo 2: Antecedentes y estado del arte.} Revisa la evolución de los sistemas basados en reglas en robótica, las características de CLIPS, los enfoques modernos de aprendizaje automático y los sistemas híbridos existentes.

    \item \textbf{Capítulo 3: Marco teórico y conceptual.} Describe los fundamentos de la planificación jerárquica, los sistemas de producción, los modelos de lenguaje grande (LLMs) y la arquitectura de integración propuesta.

    \item \textbf{Capítulo 4: Metodología.} Detalla el diseño del sistema híbrido, incluyendo la definición de reglas en CLIPS, la integración con ChatGPT/Qwen, y los mecanismos de planificación dual y validación.

    \item \textbf{Capítulo 5: Implementación.} Explica los entornos de desarrollo, la configuración técnica, la interfaz con APIs y las pruebas utilizados para validar la integración de los módulos.

    \item \textbf{Capítulo 6: Escenarios de validación y experimentación.} Presenta el diseño experimental, los escenarios de prueba y las métricas cuantitativas y cualitativas utilizadas para evaluar el sistema.

    \item \textbf{Capítulo 7: Análisis de resultados.} Compara el rendimiento del sistema híbrido frente al sistema basado solo en reglas, evalúa la efectividad de los LLMs y discute limitaciones y costos computacionales.

    \item \textbf{Capítulo 8: Discusión.} Interpreta los resultados en relación con los objetivos, analiza ventajas y desventajas del enfoque híbrido, y explora implicaciones éticas y de seguridad.

    \item \textbf{Capítulo 9: Contribuciones y relevancia.} Destaca la aportación técnica del marco híbrido, su aplicabilidad práctica en distintos sectores y su relevancia académica como benchmark.

    \item \textbf{Capítulo 10: Conclusiones y trabajo futuro.} Resume los hallazgos principales y propone líneas de investigación futura, como el fine-tuning de LLMs y la mejora de mecanismos de seguridad.
\end{enumerate}