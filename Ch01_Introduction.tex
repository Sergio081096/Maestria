\chapter{Introducción}

La robótica de servicio doméstico representa uno de los campos de mayor proyección en la inteligencia artificial y la automatización, con aplicaciones que van desde la asistencia a personas mayores hasta la gestión autónoma de entornos residenciales. En este contexto, la planificación de acciones —es decir, la capacidad de un robot para descomponer objetivos de alto nivel en secuencias ejecutables— es un desafío central, particularmente en entornos dinámicos y no estructurados donde interactúan humanos, objetos móviles y tareas imprevistas.

Los sistemas basados en reglas, como los implementados en motores de inferencia del tipo CLIPS, han demostrado ser robustos y predecibles en escenarios estructurados, como líneas de producción o laboratorios controlados. Su fortaleza radica en la transparencia del razonamiento simbólico y la capacidad de priorizar acciones críticas, como la evitación de colisiones o la gestión de emergencias. Sin embargo, su rigidez los hace poco adaptables ante variaciones no previstas en el entorno o ante comandos expresados en lenguaje natural con alto grado de ambigüedad o complejidad.

Recientemente, los modelos de lenguaje grande (LLMs), como ChatGPT o Qwen, han emergido como herramientas capaces de interpretar y generar lenguaje natural, e incluso de actuar como agentes de planificación autónomos. Su flexibilidad contextual permite traducir instrucciones verbales en secuencias de acciones, complementando así la solidez de los sistemas basados en reglas. No obstante, su naturaleza probabilística y la falta de garantías formales de seguridad limitan su uso directo en aplicaciones robóticas donde la integridad física y la predictibilidad son prioritarias.

Esta tesis propone un sistema híbrido de planificación de acciones que integra un motor de reglas CLIPS con modelos de lenguaje natural (ChatGPT/Qwen), aprovechando las ventajas de ambos enfoques: la previsibilidad y seguridad de CLIPS, y la adaptabilidad y capacidad de comprensión lingüística de los LLMs. El sistema se implementa y valida en el robot de servicio doméstico Justina, desarrollado en el Laboratorio de Bio-Robótica de la UNAM, utilizando ROS 2 como marco de integración.

\section{Contexto y Motivación}
\section{Problemática: Sistemas Puramente Basados en Reglas en Entornos Dinámicos}
\section{Hipótesis: La Complementariedad entre Sistemas Simbólicos (CLIPS) y Subsimbólicos (ChatGPT/Qwen)}
\section{Objetivos}

\begin{itemize}
    \item \textbf{Diseño del Sistema Híbrido:}
    \begin{itemize}
        \item Definir un conjunto jerárquico de reglas en CLIPS para priorizar acciones críticas.
        \item Especificar la arquitectura de integración entre CLIPS y modelos de lenguaje natural.
        \item Diseñar un marco de planificación dual que permita tanto a CLIPS como a ChatGPT generar y ejecutar planes.
        \item Establecer protocolos de comunicación y validación entre los módulos.
    \end{itemize}
    \item \textbf{Implementación Técnica:}
    \begin{itemize}
        \item Integrar el motor de reglas CLIPS con ROS (Robot Operating System).
        \item Desarrollar el módulo de interfaz con APIs de ChatGPT o implementación local de Qwen2.5-0.5B.
        \item Implementar capacidades de planificación autónoma en ChatGPT mediante prompting estructurado.
        \item Desarrollar un mecanismo de selección y validación entre planes generados por diferentes motores.
    \end{itemize}
    \item \textbf{Capacidades de Interacción Natural:}
    \begin{itemize}
        \item Implementar traducción de comandos en lenguaje natural a hechos estructurados de CLIPS.
        \item Desarrollar mecanismos de diálogo para resolución de ambigüedades.
        \item Crear protocolos de retroalimentación para aprendizaje incremental.
    \end{itemize}
    \item \textbf{Validación Experimental:}
    \begin{itemize}
        \item Validar el sistema en escenarios realistas con el robot físico y simulaciones.
        \item Evaluar el rendimiento mediante métricas comparativas:
        \begin{itemize}
            \item Tiempo de ejecución de tareas y tasa de éxito.
            \item Precisión en interpretación de comandos complejos.
            \item Robustez ante instrucciones ambiguas o novedosas.
        \end{itemize}
        \item Comparar el sistema híbrido vs. abordajes puramente basados en reglas.
        \item Evaluar la calidad de planes generados por CLIPS vs. ChatGPT en diferentes escenarios.
    \end{itemize}
\end{itemize}